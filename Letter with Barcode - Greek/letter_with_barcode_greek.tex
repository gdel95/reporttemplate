\documentclass{letter}
\usepackage[a4paper,left=2.5cm,right=2.5cm,top=3cm,twoside=true]{geometry}
\usepackage{hyperref}
\usepackage{pstricks}
\usepackage{pst-barcode}
\usepackage{color}

%Aparaitita paketa gia ellinika
\usepackage{xltxtra} 
\usepackage{xgreek} 
%I poly kali grammatoseira GFS Didot pou kanoun ta ellinika na miazoun stin grammatoseira tou LaTeX
\setmainfont[Mapping=tex-text]{GFS Didot} 

\signature{Κώστας Νικολάου}
%Sender's address and Barcode. Add or remove digits to the barcode to fit the width of the address
\address{Σμύρνης 101 \\ Παγκράτι \\ Αθήνα Τ.Κ. 12345 \\ \\ \psbarcode{12345}{}{royalmail} }

%Move the closing and signature parts to the left
\longindentation=0pt
%Remove page numbers
\pagenumbering{gobble}

\begin{document}
%Delivery address
\begin{letter}{Διεύθυνση προσωπικού \\ Γ \& Δ Α.Ε. \\ Ιερά οδός 123
\\ Αθήνα Τ.Κ. 12345}
\opening{Αξιότιμε κύρια/κυρία,}
%Dummy text
Ο Όμηρος είναι ο υποτιθέμενος συγγραφέας της Οδύσσειας (όπως και της Ιλιάδας), ο οποίος πιστεύεται οτι έζησε κατά τον 8ο ή 7ο αιώνα π.Χ. Η ύπαρξή του αμφισβητείται από κάποιους λόγιους που βλέπουν ασυμφωνίες στα δύο επικά ποιήματα, αλλά υποστηρίζεται από άλλους που βλέπουν τη συνολική συνέπεια και δηλώνουν οτι τα ποιήματα αυτά θα μπορούσαν να είναι μόνον το έργο μιας και μοναδικής ιδιοφυΐας. Αυτό που σίγουρα γνωρίζουμε είναι οτι τόσο η Ιλιάδα όσο και η Οδύσσεια υπέστησαν μια διεργασία τυποποίησης, ιδίως την εποχή του Αθηναίου τυράννου Ιππάρχου (6ος αι. π.Χ.), ο οποίος αναμόρφωσε την απαγγελία της Ομηρικής ποίησης. Σχεδόν τίποτε δεν είναι γνωστό για τον Όμηρο. Η Χίος, και αρχαίες ελληνικές πόλεις της δυτικής ακτής της Μικράς Ασίας έχουν υποστηρίξει οτι υπήρξαν γενέτειρές του.

Η Οδύσσεια αποτελείται από 24 ραψωδίες (“κεφάλαια”). Λόγω του αριθμού-τους, σε κάθε μία δίνεται το όνομα ενός από τα 24 γράμματα του ελληνικού αλφαβήτου. Αρχικά δεν υπήρχε τέτοιος διαχωρισμός σε ραψωδίες (η πρακτική αυτή εισήχθη κατά τον 3ο αι. π.Χ.), και το ποίημα δεν ήταν καν γραπτό, μόνον απαγγελόταν, και περνούσε από γενεά σε γενεά μέσω της μνήμης. Εντούτοις, το ελληνικό αλφάβητο εισήχθη κατά τον 8ο αι. π.Χ., λίγο πριν τη γέννηση του Ομήρου (ή της Οδύσσειας), άρα είναι δυνατό να καταγράφηκε το ποίημα κάποια εποχή σύντομα μετά τη δημιουργία-του.

Το ποίημα περιγράφει το δεκάχρονο ταξίδι του βασιλιά Οδυσσέα, από τα πεδία μαχών της Τροίας στο βασίλειό του στην Ιθάκη. Ξεκινά στο μέσον του ταξιδιού του Οδυσσέα, όταν ο γιος-του Τηλέμαχος ταξιδεύει σε κοντινά βασίλεια ρωτώντας να μάθει για τον πατέρα-του, ενώ ο Οδυσσέας βρίσκεται στο νησί των Φαιάκων και διηγείται τα παθήματά του στο προηγούμενο μέρος του ταξιδιού. Όμως πίσω στην Ιθάκη, μνηστήρες πλησιάζουν τη σύζυγό του, βασίλισσα Πηνελόπη, κατασπαταλώντας την περιουσία-του καθώς μένουν στο παλάτι, και προσπαθώντας να πείσουν την Πηνελόπη να διαλέξει και να παντρευτεί έναν απ’ αυτούς. Εντωμεταξύ, με τη βοήθεια των Φαιάκων και της θεάς Αθηνάς, ο Οδυσσέας ετοιμάζεται να επιστρέψει στα πατρώα εδάφη-του.
 
Σας ευχαριστώ για το χρόνο σας.
 
Είμαι στη διάθεσή σας για περισσότερες λεπτομέρειες.


 
\closing{Με εκτίμηση,}

%Postscript and attachments 
\ps{Υ.Γ. Κάτι το οποίο ξέχασα να γράψω\\}
%Remove the line bellow and replace it with just \encl{Copyright permission form} if you don't want the gray color
{\color{gray}\encl{Βιογραφικό σημείωμα}}
 
\end{letter}
 
\end{document}